\chapter{Programming - General Guidelines}
\section{Best Practices}\paragraph*{Block cut programming}can be a lengthy process, with frequent interruptions at times. Using the \textbf{CHECK PROGRAM} screens to verify where you left off is an easy way to continue programming from the correct spot. If you combine this with stroking out, or otherwise indicating completion point on your Operation Sheets, is a good double check practice to use. When starting a new program for cutting blocks with the saw, it is always a good idea to follow these few steps at first ...
\begin{list}{$\diamond$}{}
	\item \textbf{Reset Automatic Sequence}
	\item \textbf{Clear Block Program}
	\item \textbf{Home Saw}
\end{list}
This will ensure the system is ready to start "fresh" with a new program to execute (once the Operator enters it). 
\paragraph{\textbf{\LARGE \textcolor{blue}{i}}}It is also good practice to take the time during any settings changes, either program type, or operation type, to allow for the communication process between the PLC and HMI to complete. In the case of most settings there is a standard request for change then change complete response which takes place. This causes what appears to be delays from the point of view of the Operator. Although this time is merely seconds at worst, when in the "heat" of work, and pressed for time, it may seem to drag on.